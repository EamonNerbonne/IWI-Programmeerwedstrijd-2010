% ----------------------------------------------------------------------------------------
%
% Common macros for all latex files.
%
% ----------------------------------------------------------------------------------------

\pagestyle{plain}

\usepackage[english]{babel}
\usepackage[hmargin=2.5cm,vmargin=2.5cm]{geometry}
\usepackage{graphicx}
\usepackage{latexsym}
%\usepackage{comment} % uitcommentarieren van blokken oud spul
\usepackage{parskip} % niet inspringen paragrafen
\usepackage{verbatim}
\usepackage{ifthen}
\usepackage{wrapfig}
\usepackage{subfigure}
\usepackage{amsmath}
\usepackage{color}
\usepackage{fancyhdr}
\usepackage[scaled=0.92]{helvet}

% ----------------------------------------------------------------------------------------
% Utilities
% ----------------------------------------------------------------------------------------

\definecolor{darkred}{rgb}{0.5,0,0}
\newcommand{\TODO}[1]{\hspace*{0cm}\\\marginpar{\textsf{\color{red}$\leftarrow$}}\textsf{\color{red}TODO:} \textcolor{darkred}{#1}}
\newcommand{\NOTTODO}[1]{}

\newcommand{\literaloutput}[1]{``\texttt{#1}''}

\newcommand{\nonterminal}[1]{\ensuremath{\langle\text{\sc #1}\rangle}}
\newcommand{\terminal}[1]{\ensuremath{\text{``\texttt{#1}''}}}
\newcommand{\followedby}{\hspace{.5ex}}
\newcommand{\alternative}{\ \mid\ }
\newcommand{\grammarrule}{::=\ }

\newcommand\unit[1]{\,\text{#1}}
\def\mps{\,{}^\text{m}\!/\!_\text{s}} % meter/second
\def\mpss{\,{}^\text{m}\!/\!_{\text{s}^2}} % meter/second^2
\def\kg{\unit{kg}}
\def\meter{\unit{m}}

\def\StandaardZin{This year witnessed the flawless introduction of the OV-chipcard.}

% ----------------------------------------------------------------------------------------
% Config
% ----------------------------------------------------------------------------------------

\newcommand{\versionmarker}{{\small ~(version: \today)\\}}

\newcommand{\inputproblem}[1]{
  \import{../problems/#1/text/}{problem}
  \cleardoublepage
}

% used by stand-alone.tex
\def\MacrosIncluded{yes}

% Are we building the stand alone version?
\ifx\StandAlone\undefined
 \newcommand{\IfStandAlone}[2]{#2}
\else
 \newcommand{\IfStandAlone}[2]{#1}
\fi

% ----------------------------------------------------------------------------------------
% Licht/zwaar
% ----------------------------------------------------------------------------------------

\ifx\categorie\undefined
\newcommand{\categorie}{test}
\fi

% Voorbeeld: \voor{licht}{k < 1000}{k < 100000}
\newcommand{\voor}[3]{%
\ifthenelse{\equal{\categorie}{test}}{%
  \ifthenelse{\equal{#2}{}}{}{{\small\sf[#1]}#2}%
  {\small\sf[$\neg$#1]}#3%
  {\small\sf[all]}}{%
\ifthenelse{\equal{\categorie}{#1}}{#2}{#3}}%
}
\newcommand{\Voor}[3]{%
\ifthenelse{\equal{\categorie}{test}}{{\small\sf[#1]}\\ #2 \\{\small\sf[$\neg$#1]}\\ #3}{%
\ifthenelse{\equal{\categorie}{#1}}{#2}{#3}}%
}

% ----------------------------------------------------------------------------------------
% Sections
% ----------------------------------------------------------------------------------------

% counters > 26
% http://groups.google.com/group/comp.text.tex/browse_thread/thread/83116f09b1ee3e7f/17a74cd721641038?pli=1
\makeatletter
\newcommand\Aalph[1]{\expandafter\@Aalph\csname c@#1\endcsname}
\newcommand\@Aalph[1]{\@tempcnta#1%
  \ifnum\@tempcnta<27
    \@Alph\@tempcnta
  \else
    \advance\@tempcnta\m@ne
    \@tempcntb\@tempcnta
    \divide \@tempcnta by26
    \@Alph\@tempcnta
    \multiply \@tempcnta by26
    \advance \@tempcntb -\@tempcnta
    \advance \@tempcntb \@ne
    \@Alph\@tempcntb
  \fi
}
\makeatother


\newcommand{\Section}[1]{\section*{#1}}

\newcommand{\partcounter}{\Aalph{part}}
\newcommand{\Problem}[1]{{
  \refstepcounter{part}%
  \gdef\chaptername{\partcounter. #1}%
  %\thispagestyle{empty}%
  \markboth{}{}
  {\huge\bf  \IfStandAlone{$\bullet$}{\partcounter.} #1} \\[1ex]
  \versionmarker
}}
\newcommand{\Input}{
  \Section{Input}
  On the first line of the input is a positive integer, the number of test cases.
  Then for each test case:
}
\newcommand{\Output}{
  \Section{Output}
  For each test case:
}
\newcommand{\Example}{
  \Section{Example}
  \begin{tabular}{|p{0.47\linewidth}|p{0.47\linewidth}|}%
   \hline%
    \textbf{Input} & \textbf{Output} \\%
   \hline%
    \verbatiminput{../input/example.in} & \verbatiminput{../input/example.out} \\%
   \hline%
  \end{tabular}
}
\newcommand{\ExampleSpecialForLicht}{
  \Section{Example}
  \begin{tabular}{|p{0.47\linewidth}|p{0.47\linewidth}|}%
   \hline%
    \textbf{Input} & \textbf{Output} \\%
   \hline%
    \Voor{licht}{%
      \verbatiminput{../input/example.in-licht}%
    }{%
      \verbatiminput{../input/example.in}%
    } &
    \Voor{licht}{%
      \verbatiminput{../input/example.out-licht}%
    }{%
      \verbatiminput{../input/example.out}%
    }\\%
   \hline%
  \end{tabular}
}

% ----------------------------------------------------------------------------------------
% Layout
% ----------------------------------------------------------------------------------------

\pagestyle{fancy}
\fancyhead[EL]{\chaptername}
%\renewcommand{\headrulewidth}{0pt}

\setlength{\unitlength}{1mm}

\newenvironment{points}{\begin{list}{$\bullet$}{\itemsep0pt
    \parsep0pt}}{\end{list}}

% Iets minder ruimte in geneste lijsten
\makeatletter
\def\@listii{\leftmargin\leftmarginii
   \labelwidth\leftmarginii\advance\labelwidth-\labelsep
   \topsep\z@ \parsep0.4\parskip \itemsep\z@}
\makeatother


