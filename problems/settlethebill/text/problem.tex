% ----------------------------------------------------------------------------------------
%
% Latex stuff to ensure that each problem.tex file can also be built itself,
% as well as being included in larger 'competition' tex files.
%
% ----------------------------------------------------------------------------------------

\ifx\MacrosIncluded\undefined
 \documentclass[11pt,a4paper]{article}
 \def\StandAlone{yes}
 % ----------------------------------------------------------------------------------------
%
% Common macros for all latex files.
%
% ----------------------------------------------------------------------------------------

\pagestyle{plain}

\usepackage[english]{babel}
\usepackage[hmargin=2.5cm,vmargin=2.5cm]{geometry}
\usepackage{graphicx}
\usepackage{latexsym}
%\usepackage{comment} % uitcommentarieren van blokken oud spul
\usepackage{parskip} % niet inspringen paragrafen
\usepackage{verbatim}
\usepackage{ifthen}
\usepackage{wrapfig}
\usepackage{subfigure}
\usepackage{amsmath}
\usepackage{color}
\usepackage{fancyhdr}
\usepackage[scaled=0.92]{helvet}

% ----------------------------------------------------------------------------------------
% Utilities
% ----------------------------------------------------------------------------------------

\definecolor{darkred}{rgb}{0.5,0,0}
\newcommand{\TODO}[1]{\hspace*{0cm}\\\marginpar{\textsf{\color{red}$\leftarrow$}}\textsf{\color{red}TODO:} \textcolor{darkred}{#1}}
\newcommand{\NOTTODO}[1]{}

\newcommand{\literaloutput}[1]{``\texttt{#1}''}

\newcommand{\nonterminal}[1]{\ensuremath{\langle\text{\sc #1}\rangle}}
\newcommand{\terminal}[1]{\ensuremath{\text{``\texttt{#1}''}}}
\newcommand{\followedby}{\hspace{.5ex}}
\newcommand{\alternative}{\ \mid\ }
\newcommand{\grammarrule}{::=\ }

\newcommand\unit[1]{\,\text{#1}}
\def\mps{\,{}^\text{m}\!/\!_\text{s}} % meter/second
\def\mpss{\,{}^\text{m}\!/\!_{\text{s}^2}} % meter/second^2
\def\kg{\unit{kg}}
\def\meter{\unit{m}}

\def\StandaardZin{This year witnessed the flawless introduction of the OV-chipcard.}

% ----------------------------------------------------------------------------------------
% Config
% ----------------------------------------------------------------------------------------

\newcommand{\versionmarker}{{\small ~(version: \today)\\}}

\newcommand{\inputproblem}[1]{
  \import{../problems/#1/text/}{problem}
  \cleardoublepage
}

% used by stand-alone.tex
\def\MacrosIncluded{yes}

% Are we building the stand alone version?
\ifx\StandAlone\undefined
 \newcommand{\IfStandAlone}[2]{#2}
\else
 \newcommand{\IfStandAlone}[2]{#1}
\fi

% ----------------------------------------------------------------------------------------
% Licht/zwaar
% ----------------------------------------------------------------------------------------

\ifx\categorie\undefined
\newcommand{\categorie}{test}
\fi

% Voorbeeld: \voor{licht}{k < 1000}{k < 100000}
\newcommand{\voor}[3]{%
\ifthenelse{\equal{\categorie}{test}}{%
  \ifthenelse{\equal{#2}{}}{}{{\small\sf[#1]}#2}%
  {\small\sf[$\neg$#1]}#3%
  {\small\sf[all]}}{%
\ifthenelse{\equal{\categorie}{#1}}{#2}{#3}}%
}
\newcommand{\Voor}[3]{%
\ifthenelse{\equal{\categorie}{test}}{{\small\sf[#1]}\\ #2 \\{\small\sf[$\neg$#1]}\\ #3}{%
\ifthenelse{\equal{\categorie}{#1}}{#2}{#3}}%
}

% ----------------------------------------------------------------------------------------
% Sections
% ----------------------------------------------------------------------------------------

% counters > 26
% http://groups.google.com/group/comp.text.tex/browse_thread/thread/83116f09b1ee3e7f/17a74cd721641038?pli=1
\makeatletter
\newcommand\Aalph[1]{\expandafter\@Aalph\csname c@#1\endcsname}
\newcommand\@Aalph[1]{\@tempcnta#1%
  \ifnum\@tempcnta<27
    \@Alph\@tempcnta
  \else
    \advance\@tempcnta\m@ne
    \@tempcntb\@tempcnta
    \divide \@tempcnta by26
    \@Alph\@tempcnta
    \multiply \@tempcnta by26
    \advance \@tempcntb -\@tempcnta
    \advance \@tempcntb \@ne
    \@Alph\@tempcntb
  \fi
}
\makeatother


\newcommand{\Section}[1]{\section*{#1}}

\newcommand{\partcounter}{\Aalph{part}}
\newcommand{\Problem}[1]{{
  \refstepcounter{part}%
  \gdef\chaptername{\partcounter. #1}%
  %\thispagestyle{empty}%
  \markboth{}{}
  {\huge\bf  \IfStandAlone{$\bullet$}{\partcounter.} #1} \\[1ex]
  \versionmarker
}}
\newcommand{\Input}{
  \Section{Input}
  On the first line of the input is a positive integer, the number of test cases.
  Then for each test case:
}
\newcommand{\Output}{
  \Section{Output}
  For each test case:
}
\newcommand{\Example}{
  \Section{Example}
  \begin{tabular}{|p{0.47\linewidth}|p{0.47\linewidth}|}%
   \hline%
    \textbf{Input} & \textbf{Output} \\%
   \hline%
    \verbatiminput{../input/example.in} & \verbatiminput{../input/example.out} \\%
   \hline%
  \end{tabular}
}
\newcommand{\ExampleSpecialForLicht}{
  \Section{Example}
  \begin{tabular}{|p{0.47\linewidth}|p{0.47\linewidth}|}%
   \hline%
    \textbf{Input} & \textbf{Output} \\%
   \hline%
    \Voor{licht}{%
      \verbatiminput{../input/example.in-licht}%
    }{%
      \verbatiminput{../input/example.in}%
    } &
    \Voor{licht}{%
      \verbatiminput{../input/example.out-licht}%
    }{%
      \verbatiminput{../input/example.out}%
    }\\%
   \hline%
  \end{tabular}
}

% ----------------------------------------------------------------------------------------
% Layout
% ----------------------------------------------------------------------------------------

\pagestyle{fancy}
\fancyhead[EL]{\chaptername}
%\renewcommand{\headrulewidth}{0pt}

\setlength{\unitlength}{1mm}

\newenvironment{points}{\begin{list}{$\bullet$}{\itemsep0pt
    \parsep0pt}}{\end{list}}

% Iets minder ruimte in geneste lijsten
\makeatletter
\def\@listii{\leftmargin\leftmarginii
   \labelwidth\leftmarginii\advance\labelwidth-\labelsep
   \topsep\z@ \parsep0.4\parskip \itemsep\z@}
\makeatother



 \begin{document}
  % ----------------------------------------------------------------------------------------
%
% Latex stuff to ensure that each problem.tex file can also be built itself,
% as well as being included in larger 'competition' tex files.
%
% ----------------------------------------------------------------------------------------

\ifx\MacrosIncluded\undefined
 \documentclass[11pt,a4paper]{article}
 \def\StandAlone{yes}
 \input{../../../tools/macros}
 \begin{document}
  \input{problem} % hack
 \end{document}
\fi

\Problem{Door dik en dun} \TODO{Titel verengelsen}

%% author:              Marten Veldthuis
%% reviews:             -
%% text-completion:	0%
%% contest:		- (example)
%% keywords:		example
%% difficulty:		1..5

% NS poortjes beveiligen station. Mensen moeten alleen wel trein kunnen halen. Genoeg poortjes zijn dus nodig.

With the introduction of the OV-chipcard, the dutch railway company is planning to install access gates at the entrance of all their stations. These gates will prevent anyone without a chipcard from entering the station, which will obviously increase safety.

% There are to types of travellers going through this airport, the ordinary sized people, and the width challenged. Because wide gates are more expensive to purchase than normal, slim gates, the management needs you to figure out how many gates of each type they will need.

\TODO{wide+slim / large+small / normal+large?
Twan: Mij lijkt "normal" en "large" (of "extra large" of "oversized") het beste.
}

\Input
\begin{itemize}
 \item A line with a two positive integers $c_l$ and $c_x$ ($0 < c_l \le c_x < 10^9$), the cost of a large and an extra large
 gate in Euros.
 
 \item A line with a single non-negative integer $n < 10^6$, the number of travelers.
 
 \item $n$ lines, each containing a non-negative integer $t_i < 10^9$, a positive integer $d_i < 10^9$ and a string $s_i$; the arrival time (in minutes after the gates are installed), the time spent occupying the gate (in minutes), and the size of the person, either \literaloutput{L} or \literaloutput{XL}.
\end{itemize}

\Output
\begin{itemize}
 \item One line containing a single integer, the cost of the cheapest configuration of wide and slim gates in Euros.
\end{itemize}

\Example
 % hack
 \end{document}
\fi

\Problem{Settle the Bill}

%% author:			Jasper
%% reviews:         -
%% text-completion:	0%
%% contest:			iwi2010
%% keywords:		example
%% difficulty:		1..5

\def\BillHero{Professor Bakker}

{\StandaardZin}
Unfortunately the new OV-chipcard does not allow for multiple people to travel on one card.
This is bad news for the Computer Science department as they frequently travel together, sharing the costs.
They have not settled the last few travel bills yet, expecting that in the long run everything would cancel out.
However, because of the economic crisis they agree to settle the bills now, once and for all.

%As they have not settled the last few travel bills yet (hoping that in the long run it would cancel out) they agree to settle them now, once and for all.

As a certain kind of laziness comes natural to every good computer scientist: they naturally want to minimize the number of transactions.
After a few all-nighters {\BillHero} has figured out that $n-1$ is an upper bound for the number of transactions needed to settle debts between $n$ persons.
His general theory of settlements is based on the assumption that the number of transactions over $x$ days is described by %$t(x) = \frac{x^{(n-1)}-1}{x-1}$.
$t(x) = \bigl(x^{(n-1)}-1\bigr)/\bigl(x-1\bigr)$.
In the optimal case where all transactions are settled in a single day this gives $\lim_{x\rightarrow 1}\bigl(x^{(n-1)}-1\bigr)/\bigl(x-1\bigr)=n-1$.

%\NOTTODO{More math mumbo-jumbo? E.g. $\lim_{x\rightarrow 1}\frac{x^{(n-1)}-1}{x-1}=n-1$.}
Some people think they can do better though.
Can they?

\Input
\begin{itemize}
 \item A line containing two positive integers $n < 20$ and $m < 400$, the number of people and the number of debts respectively.
 \item $m$ lines containing three integers $0 \leq f_i, t_i < n$ and $a_i < 10^7$, the zero-based index of the person who owes money, the zero-based index of the person to whom he owes money, and the amount of money owed in Euros.
\end{itemize}

\Output
\begin{itemize}
 \item One line containing \literaloutput{tight} if {\BillHero}'s bound is tight (exactly $n-1$ transactions are needed) in this case, or \literaloutput{loose} if the bound is not tight (in this case).
\end{itemize}

\Example
