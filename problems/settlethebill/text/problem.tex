% ----------------------------------------------------------------------------------------
%
% Latex stuff to ensure that each problem.tex file can also be built itself,
% as well as being included in larger 'competition' tex files.
%
% ----------------------------------------------------------------------------------------

\ifx\MacrosIncluded\undefined
 \documentclass[11pt,a4paper]{article}
 \def\StandAlone{yes}
 % ----------------------------------------------------------------------------------------
%
% Common macros for all latex files.
%
% ----------------------------------------------------------------------------------------

\pagestyle{plain}

\usepackage[english]{babel}
\usepackage[hmargin=2.5cm,vmargin=2.5cm]{geometry}
\usepackage{graphicx}
\usepackage{latexsym}
%\usepackage{comment} % uitcommentarieren van blokken oud spul
\usepackage{parskip} % niet inspringen paragrafen
\usepackage{verbatim}
\usepackage{ifthen}
\usepackage{wrapfig}
\usepackage{subfigure}
\usepackage{amsmath}
\usepackage{color}
\usepackage{fancyhdr}
\usepackage[scaled=0.92]{helvet}

% ----------------------------------------------------------------------------------------
% Utilities
% ----------------------------------------------------------------------------------------

\definecolor{darkred}{rgb}{0.5,0,0}
\newcommand{\TODO}[1]{\hspace*{0cm}\\\marginpar{\textsf{\color{red}$\leftarrow$}}\textsf{\color{red}TODO:} \textcolor{darkred}{#1}}
\newcommand{\NOTTODO}[1]{}

\newcommand{\literaloutput}[1]{``\texttt{#1}''}

\newcommand{\nonterminal}[1]{\ensuremath{\langle\text{\sc #1}\rangle}}
\newcommand{\terminal}[1]{\ensuremath{\text{``\texttt{#1}''}}}
\newcommand{\followedby}{\hspace{.5ex}}
\newcommand{\alternative}{\ \mid\ }
\newcommand{\grammarrule}{::=\ }

\newcommand\unit[1]{\,\text{#1}}
\def\mps{\,{}^\text{m}\!/\!_\text{s}} % meter/second
\def\mpss{\,{}^\text{m}\!/\!_{\text{s}^2}} % meter/second^2
\def\kg{\unit{kg}}
\def\meter{\unit{m}}

\def\StandaardZin{This year witnessed the flawless introduction of the OV-chipcard.}

% ----------------------------------------------------------------------------------------
% Config
% ----------------------------------------------------------------------------------------

\newcommand{\versionmarker}{{\small ~(version: \today)\\}}

\newcommand{\inputproblem}[1]{
  \import{../problems/#1/text/}{problem}
  \cleardoublepage
}

% used by stand-alone.tex
\def\MacrosIncluded{yes}

% Are we building the stand alone version?
\ifx\StandAlone\undefined
 \newcommand{\IfStandAlone}[2]{#2}
\else
 \newcommand{\IfStandAlone}[2]{#1}
\fi

% ----------------------------------------------------------------------------------------
% Licht/zwaar
% ----------------------------------------------------------------------------------------

\ifx\categorie\undefined
\newcommand{\categorie}{test}
\fi

% Voorbeeld: \voor{licht}{k < 1000}{k < 100000}
\newcommand{\voor}[3]{%
\ifthenelse{\equal{\categorie}{test}}{%
  \ifthenelse{\equal{#2}{}}{}{{\small\sf[#1]}#2}%
  {\small\sf[$\neg$#1]}#3%
  {\small\sf[all]}}{%
\ifthenelse{\equal{\categorie}{#1}}{#2}{#3}}%
}
\newcommand{\Voor}[3]{%
\ifthenelse{\equal{\categorie}{test}}{{\small\sf[#1]}\\ #2 \\{\small\sf[$\neg$#1]}\\ #3}{%
\ifthenelse{\equal{\categorie}{#1}}{#2}{#3}}%
}

% ----------------------------------------------------------------------------------------
% Sections
% ----------------------------------------------------------------------------------------

% counters > 26
% http://groups.google.com/group/comp.text.tex/browse_thread/thread/83116f09b1ee3e7f/17a74cd721641038?pli=1
\makeatletter
\newcommand\Aalph[1]{\expandafter\@Aalph\csname c@#1\endcsname}
\newcommand\@Aalph[1]{\@tempcnta#1%
  \ifnum\@tempcnta<27
    \@Alph\@tempcnta
  \else
    \advance\@tempcnta\m@ne
    \@tempcntb\@tempcnta
    \divide \@tempcnta by26
    \@Alph\@tempcnta
    \multiply \@tempcnta by26
    \advance \@tempcntb -\@tempcnta
    \advance \@tempcntb \@ne
    \@Alph\@tempcntb
  \fi
}
\makeatother


\newcommand{\Section}[1]{\section*{#1}}

\newcommand{\partcounter}{\Aalph{part}}
\newcommand{\Problem}[1]{{
  \refstepcounter{part}%
  \gdef\chaptername{\partcounter. #1}%
  %\thispagestyle{empty}%
  \markboth{}{}
  {\huge\bf  \IfStandAlone{$\bullet$}{\partcounter.} #1} \\[1ex]
  \versionmarker
}}
\newcommand{\Input}{
  \Section{Input}
  On the first line of the input is a positive integer, the number of test cases.
  Then for each test case:
}
\newcommand{\Output}{
  \Section{Output}
  For each test case:
}
\newcommand{\Example}{
  \Section{Example}
  \begin{tabular}{|p{0.47\linewidth}|p{0.47\linewidth}|}%
   \hline%
    \textbf{Input} & \textbf{Output} \\%
   \hline%
    \verbatiminput{../input/example.in} & \verbatiminput{../input/example.out} \\%
   \hline%
  \end{tabular}
}
\newcommand{\ExampleSpecialForLicht}{
  \Section{Example}
  \begin{tabular}{|p{0.47\linewidth}|p{0.47\linewidth}|}%
   \hline%
    \textbf{Input} & \textbf{Output} \\%
   \hline%
    \Voor{licht}{%
      \verbatiminput{../input/example.in-licht}%
    }{%
      \verbatiminput{../input/example.in}%
    } &
    \Voor{licht}{%
      \verbatiminput{../input/example.out-licht}%
    }{%
      \verbatiminput{../input/example.out}%
    }\\%
   \hline%
  \end{tabular}
}

% ----------------------------------------------------------------------------------------
% Layout
% ----------------------------------------------------------------------------------------

\pagestyle{fancy}
\fancyhead[EL]{\chaptername}
%\renewcommand{\headrulewidth}{0pt}

\setlength{\unitlength}{1mm}

\newenvironment{points}{\begin{list}{$\bullet$}{\itemsep0pt
    \parsep0pt}}{\end{list}}

% Iets minder ruimte in geneste lijsten
\makeatletter
\def\@listii{\leftmargin\leftmarginii
   \labelwidth\leftmarginii\advance\labelwidth-\labelsep
   \topsep\z@ \parsep0.4\parskip \itemsep\z@}
\makeatother



 \begin{document}
  % ----------------------------------------------------------------------------------------
%
% Latex stuff to ensure that each problem.tex file can also be built itself,
% as well as being included in larger 'competition' tex files.
%
% ----------------------------------------------------------------------------------------

\ifx\MacrosIncluded\undefined
 \documentclass[11pt,a4paper]{article}
 \def\StandAlone{yes}
 \input{../../../tools/macros}
 \begin{document}
  \input{problem} % hack
 \end{document}
\fi

\Problem{The Morning Train}

%% author:          Jasper
%% reviews:         -
%% text-completion:	90%
%% contest:		    iwi2010
%% keywords:		simulate
%% difficulty:		2

{\StandaardZin}
Inspired by this success, the Dutch railway company NS is planning to install access gates at the entrance of all their stations.
These gates will prevent anyone without a chipcard from entering the station, which will obviously increase safety.
However, some people might not have a chipcard yet and may need to buy a ticket first.

Although buying a ticket can take up to thirty minutes, most people that need a ticket can do so in no more than a few minutes.
However, there are only a limited number of ticket machines, so lines inevitably form.
As people traveling by train tend to be well-mannered no one will skip the line;
when it is their turn each person will use the first available ticket machine.
A person with a ticket or an OV-chipcard can go through an access gate.
Only one person can pass an access gate at a time; doing so takes one second.
There is also a limited number of gates, so during rush hour it's still a struggle to get through.

To evaluate their new system, NS wants to simulate a typical work day where everyone needs to catch the morning train.
Anyone that has passed the access gates at eight o'clock sharp is presumed to catch the train.
The simulation starts at four in the morning, with an empty train station.

\NOTTODO{I think this can be extended to the case where the gates are for one specific platform and there are multiple times at which trains leave, assuming all persons we consider can all have all trains that leave at that platform. This shouldn't increase the complexity of the solution (much).}

\Input
\begin{itemize}
 \item A line with two positive integers $n_g, n_t < 10$, the number of access gates and ticket machines, respectively.
 
 \item A line with a single non-negative integer $n < 10^6$, the number of travelers.
 
 \item $n$ lines, each containing two non-negative integers $t_i < 10^9$ and $d_i \le 1800$, the arrival time (in seconds after four o'clock) and the time spent at a ticket machine (in seconds, zero means no ticket needed). The lines are sorted on the arrival time (in non-decreasing order). Two people with the same arrival times will get in line in the order in which they appear in the input. % Twan: Dit lijkt me niet nodig: (first in the input is first in line)
\end{itemize}

\Output
\begin{itemize}
 \item One line containing a single integer, the number of people who will miss the eight o'clock train.
\end{itemize}

\vfill % layout
\Example
 % hack
 \end{document}
\fi

\Problem{Settle the Bill}

%% author:			Jasper
%% reviews:         -
%% text-completion:	0%
%% contest:			iwi2010
%% keywords:		example
%% difficulty:		1..5

\def\BillHero{Professor Bakker}

{\StandaardZin}
Unfortunately the new OV-chipcard does not allow for multiple people to travel on one card.
This is bad news for the Computer Science department as they frequently travel together, sharing the costs.
They have not settled the last few travel bills yet, expecting that in the long run everything would cancel out.
However, because of the economic crisis they agree to settle the bills now, once and for all.

%As they have not settled the last few travel bills yet (hoping that in the long run it would cancel out) they agree to settle them now, once and for all.

As a certain kind of laziness comes natural to every good computer scientist: they naturally want to minimize the number of transactions.
After a few all-nighters {\BillHero} has figured out that $n-1$ is an upper bound for the number of transactions needed to settle debts between $n$ persons.
His general theory of settlements is based on the assumption that the number of transactions over $x$ days is described by %$t(x) = \frac{x^{(n-1)}-1}{x-1}$.
$t(x) = \bigl(x^{(n-1)}-1\bigr)/\bigl(x-1\bigr)$.
In the optimal case where all transactions are settled in a single day this gives $\lim_{x\rightarrow 1}\bigl(x^{(n-1)}-1\bigr)/\bigl(x-1\bigr)=n-1$.

%\NOTTODO{More math mumbo-jumbo? E.g. $\lim_{x\rightarrow 1}\frac{x^{(n-1)}-1}{x-1}=n-1$.}
Some people think they can do better though.
Can they?

\Input
\begin{itemize}
 \item A line containing two positive integers $n < 20$ and $m < 400$, the number of people and the number of debts respectively.
 \item $m$ lines containing three integers $0 \leq f_i, t_i < n$ and $a_i < 10^7$, the zero-based index of the person who owes money, the zero-based index of the person to whom he owes money, and the amount of money owed in Euros.
\end{itemize}

\Output
\begin{itemize}
 \item One line containing \literaloutput{tight} if {\BillHero}'s bound is tight (exactly $n-1$ transactions are needed) in this case, or \literaloutput{loose} if the bound is not tight (in this case).
\end{itemize}

\Example
