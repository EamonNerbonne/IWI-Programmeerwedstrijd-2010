% ----------------------------------------------------------------------------------------
%
% Latex stuff to ensure that each problem.tex file can also be built itself,
% as well as being included in larger 'competition' tex files.
%
% ----------------------------------------------------------------------------------------

\ifx\MacrosIncluded\undefined
 \documentclass[11pt,a4paper]{article}
 \def\StandAlone{yes}
 % ----------------------------------------------------------------------------------------
%
% Common macros for all latex files.
%
% ----------------------------------------------------------------------------------------

\pagestyle{plain}

\usepackage[english]{babel}
\usepackage[hmargin=2.5cm,vmargin=2.5cm]{geometry}
\usepackage{graphicx}
\usepackage{latexsym}
%\usepackage{comment} % uitcommentarieren van blokken oud spul
\usepackage{parskip} % niet inspringen paragrafen
\usepackage{verbatim}
\usepackage{ifthen}
\usepackage{wrapfig}
\usepackage{subfigure}
\usepackage{amsmath}
\usepackage{color}
\usepackage{fancyhdr}
\usepackage[scaled=0.92]{helvet}

% ----------------------------------------------------------------------------------------
% Utilities
% ----------------------------------------------------------------------------------------

\definecolor{darkred}{rgb}{0.5,0,0}
\newcommand{\TODO}[1]{\hspace*{0cm}\\\marginpar{\textsf{\color{red}$\leftarrow$}}\textsf{\color{red}TODO:} \textcolor{darkred}{#1}}
\newcommand{\NOTTODO}[1]{}

\newcommand{\literaloutput}[1]{``\texttt{#1}''}

\newcommand{\nonterminal}[1]{\ensuremath{\langle\text{\sc #1}\rangle}}
\newcommand{\terminal}[1]{\ensuremath{\text{``\texttt{#1}''}}}
\newcommand{\followedby}{\hspace{.5ex}}
\newcommand{\alternative}{\ \mid\ }
\newcommand{\grammarrule}{::=\ }

\newcommand\unit[1]{\,\text{#1}}
\def\mps{\,{}^\text{m}\!/\!_\text{s}} % meter/second
\def\mpss{\,{}^\text{m}\!/\!_{\text{s}^2}} % meter/second^2
\def\kg{\unit{kg}}
\def\meter{\unit{m}}

\def\StandaardZin{This year witnessed the flawless introduction of the OV-chipcard.}

% ----------------------------------------------------------------------------------------
% Config
% ----------------------------------------------------------------------------------------

\newcommand{\versionmarker}{{\small ~(version: \today)\\}}

\newcommand{\inputproblem}[1]{
  \import{../problems/#1/text/}{problem}
  \cleardoublepage
}

% used by stand-alone.tex
\def\MacrosIncluded{yes}

% Are we building the stand alone version?
\ifx\StandAlone\undefined
 \newcommand{\IfStandAlone}[2]{#2}
\else
 \newcommand{\IfStandAlone}[2]{#1}
\fi

% ----------------------------------------------------------------------------------------
% Licht/zwaar
% ----------------------------------------------------------------------------------------

\ifx\categorie\undefined
\newcommand{\categorie}{test}
\fi

% Voorbeeld: \voor{licht}{k < 1000}{k < 100000}
\newcommand{\voor}[3]{%
\ifthenelse{\equal{\categorie}{test}}{%
  \ifthenelse{\equal{#2}{}}{}{{\small\sf[#1]}#2}%
  {\small\sf[$\neg$#1]}#3%
  {\small\sf[all]}}{%
\ifthenelse{\equal{\categorie}{#1}}{#2}{#3}}%
}
\newcommand{\Voor}[3]{%
\ifthenelse{\equal{\categorie}{test}}{{\small\sf[#1]}\\ #2 \\{\small\sf[$\neg$#1]}\\ #3}{%
\ifthenelse{\equal{\categorie}{#1}}{#2}{#3}}%
}

% ----------------------------------------------------------------------------------------
% Sections
% ----------------------------------------------------------------------------------------

% counters > 26
% http://groups.google.com/group/comp.text.tex/browse_thread/thread/83116f09b1ee3e7f/17a74cd721641038?pli=1
\makeatletter
\newcommand\Aalph[1]{\expandafter\@Aalph\csname c@#1\endcsname}
\newcommand\@Aalph[1]{\@tempcnta#1%
  \ifnum\@tempcnta<27
    \@Alph\@tempcnta
  \else
    \advance\@tempcnta\m@ne
    \@tempcntb\@tempcnta
    \divide \@tempcnta by26
    \@Alph\@tempcnta
    \multiply \@tempcnta by26
    \advance \@tempcntb -\@tempcnta
    \advance \@tempcntb \@ne
    \@Alph\@tempcntb
  \fi
}
\makeatother


\newcommand{\Section}[1]{\section*{#1}}

\newcommand{\partcounter}{\Aalph{part}}
\newcommand{\Problem}[1]{{
  \refstepcounter{part}%
  \gdef\chaptername{\partcounter. #1}%
  %\thispagestyle{empty}%
  \markboth{}{}
  {\huge\bf  \IfStandAlone{$\bullet$}{\partcounter.} #1} \\[1ex]
  \versionmarker
}}
\newcommand{\Input}{
  \Section{Input}
  On the first line of the input is a positive integer, the number of test cases.
  Then for each test case:
}
\newcommand{\Output}{
  \Section{Output}
  For each test case:
}
\newcommand{\Example}{
  \Section{Example}
  \begin{tabular}{|p{0.47\linewidth}|p{0.47\linewidth}|}%
   \hline%
    \textbf{Input} & \textbf{Output} \\%
   \hline%
    \verbatiminput{../input/example.in} & \verbatiminput{../input/example.out} \\%
   \hline%
  \end{tabular}
}
\newcommand{\ExampleSpecialForLicht}{
  \Section{Example}
  \begin{tabular}{|p{0.47\linewidth}|p{0.47\linewidth}|}%
   \hline%
    \textbf{Input} & \textbf{Output} \\%
   \hline%
    \Voor{licht}{%
      \verbatiminput{../input/example.in-licht}%
    }{%
      \verbatiminput{../input/example.in}%
    } &
    \Voor{licht}{%
      \verbatiminput{../input/example.out-licht}%
    }{%
      \verbatiminput{../input/example.out}%
    }\\%
   \hline%
  \end{tabular}
}

% ----------------------------------------------------------------------------------------
% Layout
% ----------------------------------------------------------------------------------------

\pagestyle{fancy}
\fancyhead[EL]{\chaptername}
%\renewcommand{\headrulewidth}{0pt}

\setlength{\unitlength}{1mm}

\newenvironment{points}{\begin{list}{$\bullet$}{\itemsep0pt
    \parsep0pt}}{\end{list}}

% Iets minder ruimte in geneste lijsten
\makeatletter
\def\@listii{\leftmargin\leftmarginii
   \labelwidth\leftmarginii\advance\labelwidth-\labelsep
   \topsep\z@ \parsep0.4\parskip \itemsep\z@}
\makeatother



 \begin{document}
  % ----------------------------------------------------------------------------------------
%
% Latex stuff to ensure that each problem.tex file can also be built itself,
% as well as being included in larger 'competition' tex files.
%
% ----------------------------------------------------------------------------------------

\ifx\MacrosIncluded\undefined
 \documentclass[11pt,a4paper]{article}
 \def\StandAlone{yes}
 \input{../../../tools/macros}
 \begin{document}
  \input{problem} % hack
 \end{document}
\fi

\Problem{Door dik en dun} \TODO{Titel verengelsen}

%% author:              Marten Veldthuis
%% reviews:             -
%% text-completion:	0%
%% contest:		- (example)
%% keywords:		example
%% difficulty:		1..5

% NS poortjes beveiligen station. Mensen moeten alleen wel trein kunnen halen. Genoeg poortjes zijn dus nodig.

With the introduction of the OV-chipcard, the dutch railway company is planning to install access gates at the entrance of all their stations. These gates will prevent anyone without a chipcard from entering the station, which will obviously increase safety.

% There are to types of travellers going through this airport, the ordinary sized people, and the width challenged. Because wide gates are more expensive to purchase than normal, slim gates, the management needs you to figure out how many gates of each type they will need.

\TODO{wide+slim / large+small / normal+large?
Twan: Mij lijkt "normal" en "large" (of "extra large" of "oversized") het beste.
}

\Input
\begin{itemize}
 \item A line with a two positive integers $c_l$ and $c_x$ ($0 < c_l \le c_x < 10^9$), the cost of a large and an extra large
 gate in Euros.
 
 \item A line with a single non-negative integer $n < 10^6$, the number of travelers.
 
 \item $n$ lines, each containing a non-negative integer $t_i < 10^9$, a positive integer $d_i < 10^9$ and a string $s_i$; the arrival time (in minutes after the gates are installed), the time spent occupying the gate (in minutes), and the size of the person, either \literaloutput{L} or \literaloutput{XL}.
\end{itemize}

\Output
\begin{itemize}
 \item One line containing a single integer, the cost of the cheapest configuration of wide and slim gates in Euros.
\end{itemize}

\Example
 % hack
 \end{document}
\fi

\Problem{Dining Philosophers}

%% author:			Twan
%% reviews:         -
%% text-completion:	50%
%% contest:			iwi2010
%% keywords:		inkopper
%% difficulty:		0

{\StandaardZin}
Word of this has even traveled beyond this life and into the next.

The first circle of hell is reserved for the unbaptised and the virtuous pagans.
Among them are most of history's famous philosophers.
% Zie: http://en.wikipedia.org/wiki/Inferno_%28Dante%29#First_Circle_.28Limbo.29

So it comes to pass that
Immanuel Kant,
Socrates,
John Locke,
Ren\'e Descartes and
S{\o}ren Kierkegaard
decide to hold a dinner party, to ponder these new troublesome developments in the Netherlands.
They have seated themselves at a round table.
In front of each philosopher is a dinner plate, between each pair of adjacent plates is a fork.
In the center of the table is a large tray of chicken drumsticks, freshly roasted above the fires of hell.

These philosophers are among the most civilised men in history,
but even the most decent folks soon degrade to a band of complete idiots at the sight of chicken drumsticks.
To prevent this horrible outcome the philosophers have decided on a strict protocol for the dinner.
Before making any move on the delicious food a philosopher must hold both of the forks next to his plate.
Only then may he pick up and eat exactly one drumstick from the tray.
After eating this drumstick, which takes approximately 1 minute,
%the philosopher is to put down both forks to take a sip of wine, which takes about 10 seconds.
the philosopher is to put down both forks and has the opportunity to share some deep insight with the other participants, which takes only 10 seconds.
After that he may try to lift his forks again to resume eating.

% Deadlock

Upon agreeing on the protocols each of the philosophers immediately picks up the fork at his right hand side,
hoping to get the opportunity to impress the others with his wisdom (not to mention being able to savour the delicious meat he has been staring at for some time now). The philosophers quickly realise the gravity of their dilemma: their protocol is clearly flawed, yet none of them feels compelled to give up their chance to eat.

% Kant
Kant decides that it is his categorical imperative to do something about this situation.
\NOTTODO{categorical imperative is eigenlijk iets anders. Moeten we dat fixen?}
He puts down the fork in his right hand, to give someone else a chance to eat.
Socrates, who sits next to Kant, immediately takes advantage of this oportunity and picks up the fork.
Since he now holds two forks he takes a single drumstick and eats it.

% Greedy

\Input
\begin{itemize}
 \item A line with a single positive integer $i$, $i < 10^6$, the number of drumsticks on the table.
\end{itemize}

\Output
\begin{itemize}
 \item One line containing a single integer, the number of drumsticks Immanuel Kant eats.
\end{itemize}

\vfill % layout
\Example
