\documentclass[a4paper]{letter}
\usepackage[dutch]{babel}
\pagestyle{headings}
\addtolength{\textheight}{2cm}
%\signature{Twan van Laarhvoen}
%\telephone{3633971}
%\email{T.M.van.Laarhoven@rug.nl}


\begin{document}
\begin{letter}{Onderwijsinstituut Informatica en Congnitie\\
               t.a.v. Prof. dr. G.R. Renardel de Lavalette}% (adj. dir.)}
\opening{Geachte directie,\\beste Gerard}
%
%
Ook dit jaar is het weer gelukt een aantal mensen bijeen te brengen
die bereid zijn om een programmeerwedstrijd te organiseren.
De bedoeling is om deze wedstrijd plaats te laten vinden op
zaterdag 22 mei a.s.\ van 13.00 uur tot 18.00 uur.
Er zal worden gestreden in drie categori\"en: Licht (voor studenten
die in het eerste jaar zitten) en Zwaar (voor overige
studenten) en Senior (voor stafleden en alumni).
Dit jaar hebben we de opgaven in het Engels geformuleerd zodat ook buitenlandse studenten en stafleden deel kunnen nemen.

Voor deze activiteit vragen we het Onderwijsinstituut een (reguliere)
bijdrage van 300 euro.

Evenals de afgelopen jaren willen we de prijzen laten bestaan uit een
wisselbeker voor het team dat het hoogst eindigt en een aantal
geldbedragen (50 euro voor de winnaars; 35 euro voor het tweede team
en 25 euro voor het derde).  Dit voor de categorie\"en licht en zwaar.
Daarnaast krijgt elke deelnemers na de wedstrijd een vaantje
uitgereikt.

Van elk team vragen we 8 euro inschrijfgeld.

%\newpage
Onderstaande tabel geeft een begroting, uitgaande van 12 deelnemende
teams.
\begin{center}
\begin{tabular}{l r || l r}
inkomsten   &   & uitgaven & \\
\hline
inschrijfgelden    &  96           & prijzengeld          & 220 \\
subsidie OIC       & 300           & vaantjes / graveren  & 175 \\
%                   &               & org. benodigdheden   &  65\\[1ex]
totaal             & 396           & totaal               & 395 \\
\hline

\end{tabular}
\end{center}
\strut \\[3ex]
Graag horen we binnenkort de beslissing.
\strut \\[3ex]
Met vriendelijke groet,
\\[7ex]
Twan van Laarhoven\\
T.M.van.Laarhoven@student.rug.nl
%\closing{Groetend,}
\end{letter}
\end{document}
